\documentclass[a4paper, 11pt]{report}
\usepackage[margin=2.5cm]{geometry}
\usepackage[T1]{fontenc}
\usepackage[utf8]{inputenc}
\usepackage[french]{babel}
\usepackage{tikz-uml}
\usepackage{graphicx}
\usepackage{caption}
\DeclareCaptionLabelFormat{cont}{#1~#2\alph{ContinuedFloat}}
\captionsetup[ContinuedFloat]{labelformat=cont}
\usepackage{tikz}
\usetikzlibrary{snakes,arrows,shapes,automata}
\usepackage{listings}
\lstset {
  basicstyle=\footnotesize,
  tabsize=2
}



\title{Implémentation d'une API HTTP de style REST pour Apache OFBiz}

\author{Mathieu Lirzin}

\date{30 août 2018}

\begin{document}

\maketitle

\begin{abstract}
  ...
\end{abstract}

\tableofcontents
\clearpage
\chapter*{Introduction}

Dans le cadre d'une amélioration du progiciel de gestion intégré
Apache OFBiz nous souhaitons mettre en place une API HTTP basée sur le
style d'architecture REST pour les échanges entre les différents
composants logiciels de type service. Il sera demandé au stagiaire
dans un premier temps de réaliser un état de l'art de l'utilisation de
ce style architectural, puis d'étudier les principes d'architectures
orientées services (SOA) utilisés dans le framework Apache OFBiz.

Après cette phase d'analyse, l'API HTTP basée sur le style REST devra
être implementée dans le framework Apache OFBiz en suivant les
principes SOA.  Ces améliorations devront être réalisées en
interaction avec les projets clients et la communauté Apache OFBiz
afin de permettre leurs intégration dans le projet officiel.

\chapter{Présentation de l'entreprise}

Néréide est un société de service en logiciel libre fondée en 2004
spécialisée dans l'intégration du progiciel de gestion intégré Apache
OFBiz.  Il s'agit d'une société coopérative et participative (SCOP)
dont le siège social est situé à Tours.  Les activités de la société
se décompose en différents axes:

\begin{description}
\item[Dévéloppement spécifique] L'intégration du framework OFBiz
  implique généralement des développement spécifique pour s'adapter au
  différents composants du système d'information existant.
\item[Maintenance et support applicatif (TMA)] Prestation de support
  et de maintenance corrective et/ou evolutive pour les développement
  spécifiques précédemment fait
\item[Administration système] mettre en place un élément système, un
  réseau, intégrer et migrer des composant systèmes.
\end{description}

Néréide s'appuie sur le réseau Libre Entreprise dont elle fait partie
pour proposer des services tels que l'hébergement, de formation.
L'équipe est actuellement composée de 11 personnes, ayant pour point
commun une expertise fonctionnelle sur OFBiz.

\chapter{Présentation de OFBiz}

Apache \emph{Open For Business} (OFBiz) est un framework Web ayant
pour but de faciliter la construction de progiciels de gestion
intégré.  Cet objectif est réalisé au moyen d'un modèle de donnée
générique et d'un ensemble de modules dédiés à des tâches commune à la
plupart des entreprises, telles que la comptabilité, la facturation,
ou la gestion des stocks.

OFBiz a été créé en 2001 par David E. Jones et Andy Zeneski dans
l'idée de développer une solution commune aux différentes
problématiques rencontrées par les entreprises en utilisant un modèle
de développement en logiciel libre. En janvier 2006 OFBiz a été accepté
dans l'incubateur de la fondation Apache avant d'être accepté en tant
que projet a part entière en décembre 2006.

\section{Architecture Orientée Service}

OFBiz est basé sur une architecture orientée service (SOA).  La notion
de moteur de service est définit par l'interface Java
\verb=GenericEngine=.

Un service peut être appelé en lui passant un ensemble de valeur en
entrée, il renvoie alors un ensemble de valeurs en sortie.  Il est
possible d'exécuter un service soit de manière synchrone soit de
manière asynchrone.  Il peut être implémenté dans n'importe quel
langage de programmation dans la mesure ou les paramètres et résultats
doivent être sérialisable.  Les services peuvent être ordonnancé.  Ils
peuvent appeler d'autres services.

On parle d'événements OFBiz pour les requêtes HTTP envoyés au serveur
OFBiz, mais aussi d'événement de service qui correspondent à
différentes phase de l'exécution d'un service.  Les différentes phases
existantes sont "auth", "in-validate", "out-validate", "invoke",
"commit", or "return".

L'ensemble des routes sont définit dans un fichier
\verb=controller.xml=.  Seule les requêtes \verb=POST= et \verb=GET=
sont traités du fait que le client est uniquement prévu pour être un
navigateur.  Il n'y a de différences dans le traitement des requêtes.
Peu importe la méthode choisi cela à pour effet d'appeler l'evénement
associé.

\chapter{État de l'art}

\section{Representational State Tranfer}

\emph{Representational State Transfer} (REST) est un style
architectural ayant émergé au début de l'année 1995.  Il s'agit d'un
modèle idéalisé décrivant comment le \emph{Web} devrait fonctionner
\cite{fielding2002principled}.  Il a été affiné pendant 5 ans jusqu'à
être décrit en détails par Roy Fielding dans sa thèse de doctorat
\cite{fielding2000architectural} en 2000.  Un style architectural est
un ensemble de contraintes placées sur une architecture logicielle
c'est à dire une abstraction décrivant les composants, les
connecteurs, ainsi que les données présents au sein d'un système
\cite{perry1992foundations}.

\begin{figure}
  \centering
  \resizebox {0.8\columnwidth} {!} {
  \begin{tikzpicture}
    \tikzstyle{as}= [draw,rectangle,rounded corners=3pt]
    \tikzstyle{asc}= [draw,rectangle,rounded corners=3pt, fill=black!10]
    \tikzstyle{cst}= [right,scale=0.7]
    \node[as] (o) at (0, 10) [circle,fill] {};
    \node[as] (RR) at (-5, 8) {RR};
    \node[asc] (cache) at (-5, 6) {\$};

    \node[asc] (CS) at (-3, 8) {CS};
    \node[asc] (CSS) at (-3, 6) {CSS};
    \node[as] (CcacheSS) at (-3, 4) {C\$SS};

    \node[asc] (LS) at (0, 8) {LS};
    \node[as] (LCS) at (0, 6) {LCS};
    \node[as] (LCcacheSS) at (0, 4)  {LC\$SS};

    \node[as] (VM) at (2.8, 8) {VM};
    \node[asc] (COD) at (2.8, 6) {COD};
    \node[as] (LCODCcacheSS) at (2.8, 4) {LCODC\$SS};

    \node[asc] (U) at (5, 8)  {U};
    \node[as] (REST) at (5, 4) {REST};

    \draw[->] (o) edge [bend right=15] node[cst, left]{replicated} (RR);
    \draw[->] (o) edge [bend right=10] node[cst]{separated} (CS);
    \draw[->] (o) edge node[cst]{layered} (LS);
    \draw[->] (o) edge [bend left=2] node[cst]{programmable} (VM);
    \draw[->] (o) edge [bend left=15] node[cst,yshift=10]{uniform interface} (U);
    \draw[->] (RR) edge node[cst]{on-demand} (cache);
    \draw[->] (cache) edge node[cst, left]{cacheable} (CcacheSS);
    \draw[->] (CS) edge node[cst]{stateless} (CSS);
    \draw[->] (CSS) edge node[cst]{reliable} (CcacheSS);
    \draw[->] (LS) edge (LCS);
    \draw[->] (LCS) edge node[cst]{shared} (LCcacheSS);
    \draw[->] (VM) edge (COD);
    \draw[->] (COD) edge node[cst]{extensible} (LCODCcacheSS);
    \draw[->] (U) edge node[cst, text width=3cm, left,xshift=25, yshift=30]{simple visible reusable} (REST);
    \draw[->] (CS.east) edge node[cst, sloped, text
      width=2cm, centered]{intermediate processing} (LCS);
    \draw[->] (CS.east) edge[bend left=15, pos=0.75] node[cst]{mobile} (COD);
    \draw[->] (CcacheSS) edge node[cst,centered, yshift=10]{scalable} (LCcacheSS);
    \draw[->] (LCcacheSS) edge node[cst,left, text width=0.5cm]{multi-org} (LCODCcacheSS);
    \draw[->] (LCODCcacheSS) edge  (REST);
  \end{tikzpicture}
}
  \caption{Dérivation du style architectural REST}
\end{figure}

REST est composé des 6 contraintes suivantes:

\begin{description}
\item[Client-Serveur (CS)] Cette contrainte permet la séparation des
  préoccupations entre le(s) serveur(s) et le(s) client(s).
  Typiquement il est important de ne pas lier une interface
  utilisateur aux services rendus par le serveur, pour permettre la
  réutilisabilité des services et la variété des interfaces
  utilisateurs.  Dans un contexte distribué cela permet au client et
  au serveur d'évoluer indépendamment.

\item[Serveur sans état (CSS)] Chaque requête envoyé par le client
  doit pouvoir être traitée de manière indépendante.  Cela impose de
  transmettre dans chacune des requêtes les données permettant
  d'authentifier le client.  Cela rend l'architecture robuste dans la
  mesure où cela limite les contraintes d'ordre dans l'envoi de
  messages.  Dans un contexte avec plusieurs serveurs cela permet de
  passer simplement à l'échelle.  Cependant cette redondance
  d'informations entre les requêtes impose un surcout de transfert de
  données.

\item [Possible mise en cache (\$)] Pour contrecarrer ce sur-coût,
  REST impose de pouvoir spécifier dans les messages si les
  informations qu'ils contiennent peuvent être garder en cache et
  ainsi limiter le nombre de messages transmis sur le réseau.

\item[Interface uniforme (U)] Cette contrainte est centrale à REST et
  consiste dans la manipulation des ressources uniquement par leur(s)
  représentation(s), l'utilisation de messages auto-descriptif, et
  l'utilisation de \emph{l'hypermédia comme moteur de l'état de
    l'application}\footnote{plus connu sous sa forme anglaise
    \emph{Hypermedia As The Engine Of Application State} (HATEOAS)}

\item[En couches (LS)] Cette contrainte permet de limiter la
  complexité de l'architecture en revanche elle a l'inconvénient
  d'avoir un surcôut en terme d'exécution.

\item[Code à la demande (COD)] Ceci est une contrainte optionelle
  permettant au serveur d'étendre les fonctionnalités du client.
\end{description}

\section{Architecture Web}

L'architecture Web est une instance du style architectural REST.  En
fait REST a été conceptualisé conjointement au World Wide Web pour
permettre de décrire de manière abstraite les enjeux d'un tel système.
Nous présentons ici les différents composants que sont le protocole
HTTP, les URI, et les formats de données tels HTML, XML, et JSON.

\subsection{Protocole HTTP}

Le protocole \emph{Hypertext Transfer Protocol} (HTTP) est un
protocole texte utilisé sur le Web pour transmettre les ressources
HTML, CSS, Javascript.  Ce protocole suit les contraintes définies par
le style architectural REST.  Il a un rôle particulier dans
l'architecture \emph{Web} puisqu'il sert à la fois à la communications
entre les composants \emph{Web} et est l'unique protocole intégrant la
notion de représentations de ressources.  L'accès aux ressources se
fait au moyen de 4 méthodes principales \verb=GET=, \verb=POST=,
\verb=HEAD=, \verb=DELETE= mais également de 5 autres méthodes moins
courantes \verb=OPTIONS=, \verb=CONNECT=, \verb=TRACE=, \verb=PUT=,
\verb=PATCH=.

\begin{verbatim}
GET /hello.txt HTTP/1.1
Host: www.example.com
Connection: upgrade
Upgrade: HTTP/2.0, SHTTP/1.3, IRC/6.9, RTA/x11
\end{verbatim}

La standardisation des types de méthodes permettent à des clients
génériques de faire des suppositions sur le propriétés de ces
méthodes.  Par exemple de la méthode \verb=GET= est nullpotent c'est à
dire qu'elle ne doit pas modifier l'état des ressources, et la méthode
\verb=POST= est idempotent c'est à dire que la réception de plusieurs
message ne doit produire qu'une seule action.

l'URI ne doit pas être l'API mais la représentation des données
identifié par cette URI.  En pratique cela qu'il faut privélégier la
récupération d'URL dans les payloads plutôt que concaténer des chaînes
de caractères qui est un signe que l'URL est l'API.
%% Comment traiter le versioning?

Les méchanismes de cache à différents points du réseau
\cite{fielding1999hypertext}.

\subsection{Uniform Resource Identifier}

Les \emph{Uniform Resource Identifiers} (URI) servent à nommer toute
sorte des ressources (document, images, service, ...)
\cite{masinter2005uniform}.  Les URI ne sont pas des emplacements et
donc ne sont pas liées à un serveur spécifique ce qui permet
d'utiliser des proxies.  Comme montré par la <<rep-res,figure 2>> les
URI permettent uniquement d'accéder à des représentations et non
directement à des ressources.  Cette distinction permet une forme de
polymorphisme en associant plusieurs représentations à une même
ressource.  En pratique l'accès à ces représentations se fait au moyen
du header \verb=Accept= présent dans la requête émise vers une URI,
qui dispatche alors vers la représentation la plus adaptée.

\begin{center}
  \begin{tikzpicture}
    \tikzstyle{n}= [draw,rectangle,rounded corners=3pt]
    \node[n, circle] (c) at (0, 0) {Client};
    \node[n] (rep) at (4, 0) {Représentation};
    \node[n] (res) at (9, 0) {Ressource};
    \draw[->] (c) edge [bend left] node [above] {envoit} (rep);
    \draw[->] (rep) edge [bend left] node [below] {recoit} (c);
    \draw[<->] (rep) edge node [above] {met à jour} (res);
  \end{tikzpicture}
\end{center}

Les URI peuvent à la fois référencer des ressources statiques ou des
ressources dynamiques.  Une ressource $R$ est un fonction $M_r(t)$
associant à un temps $t$ une entité ou valeurs.  L'ensemble des
entités ou valeurs appartenant au codomaine sont équivalentes en terme
de leur sémantique.  Même si ces ressources pointent sur la même
valeur il s'agit bien de 2 ressources distinctes.  Ce qui identifie
une ressource c'est la sémantique associé à son URI.

\subsection{Formats de données}

Lors d'une requête, il est possible pour le client de spécifier le
type contenu souhaité au moyen du header \verb=Accept= cependant il
faut que le serveur recevant cette requête soit en mesure de fournir
le format de donnée souhaité.  Dans la pratique les formats XML est
JSON sont les plus courant pour la transmission de données
structurées.

\subsection{Hypermédia}

La notion d'\emph{hypermédia} est cruciale dans la réalisation d'une
API RESTful bien qu'elle soit trop souvent négligé.  Cette notion est
une extension du contexte d'\emph{hypertext} au formats d'image, de
son, et de vidéos.  Il désigne un réseau d'information accessible de
manière non-linéaire et interactive au moyen de liens.
%% Hypermedia is defined by the presence of application control information embedded within, or as a layer above , the presentation of information  <<modern-web-archi>>.

\subsection{Bénéfices}

Dans le contexte du \emph{Web} l'hypermédia a été choisi pour sa
simplicité et généricité.  Cela a été fait dans l'objectif de rendre
la participation au \emph{Web} accessible au plus grand nombre.
L'hypermedia permet également de stocker les différentes informations
liées sur des serveurs différents et permet donc au \emph{Web} d'être
distribué.

Découvrabilité, faible couplage, mise à jour simplifié.

\subsection{Problèmes récurrents}

- Accès aux ressources par concaténation de chaines de caractères

\chapter{Choix techniques}

\section{Propositions passées}

Cela fait plusieurs années que l'intégration d'un moyen d'accéder aux
services et entitées au moyen d'une API HTTP de style REST est en
discussion au sein de la communauté des développeurs de OFBiz.  En
effet c'est en 2011 qu'Adrian Crum proposa un premier
prototype\footnote{https://issues.apache.org/jira/browse/OFBIZ-4274}. La
première idée était de permettre de faire du \emph{CRUD} sur les
entités aux travers des méthodes HTTP.  Il a été ensuite question de
permettre d'appeler des services et de faire correspondre le
dictionnaire d'entrée et de sorties du service avec les
représentations JSON et XML.  Cependant la conversion des codes
d'erreurs d'appel de services vers des codes d'erreur HTTP semble
difficilement automatisable.  En effet les types d'errreur de services
sont \emph{Success}, \emph{Fail}, et \emph{Error}.  Lorsque l'appel
d'un service échoue du fait de la non-authentification de l'appelant
le code de retour est \emph{Fail} ce qui implique qu'il ne peut être
distinguer d'une erreur d'exécution et que donc il n'y a pas de moyen
de mapper l'échec d'authentification vers le code d'erreur 401  et

Pour satisfaire la contraintes d'interface uniforme de REST Il est
nécessaire d'implémenter l'hypermédia pour rendre les services
découvrables. Pour cela il a été proposé de combiner la méthode
\verb=OPTIONS= pour décrire les actions possibles sur une ressource au
moyen du header HTTP \verb=Allow= listant l'ensemble des Méthodes
supportées avec des URIs dans le corps de la réponses.

\begin{verbatim}
$ curl -i -X OPTIONS http://example.com/resources/1
HTTP/1.1 200 OK
Allow: GET, HEAD, POST, DELETE, [...]
Content-Length: 0
\end{verbatim}

Actuellement OFBiz gère les requêtes HTTP grâce à une servlet Java
\cite{Chan2017servlet} définit dans la classe \verb=controlServlet=.
Le traitement des requêtes HTTP considère les méthode \verb=GET= et
\verb=POST= sans disctinction.

En parallèle de cette discussion théorique et sur le language
spécifique à utiliser, il a été également question d'utiliser certains
framework pour accomplir cette tâche. Parmi ces frameworks ont été
cité:

\begin{itemize}
\item JAX-RS
\item Vert.x
\item Apache Juneau
\item Swagger.IO
\end{itemize}

\subsection{JAX-RS}

\emph{JAX-RS} est une API java permettant de créer des services Web «
RESTful » \cite{pericas2013jax}. Cette spécification est intégrée dans
\emph{JavaEE} et possède plusieurs implémentations dont Jersey qui est
l'implémentation de référence, RESTeasy, Restlet, et Apache CXF.
L'utilisation de Jersey est impossible du fait de l'incompatibilité de
sa license avec celle d'OFBiz.  Du fait de son intégration dans la
communauté Apache, CXF semble être une implémentation adapté à une
utilisation dans OFBiz.  L'avantage de cette solution est une
intégration et adoption forte dans l'univers JavaEE.  Du fait qu'OFBiz
est déjà couplé au JDK ce choix ne présente pas un grand risque.

Cette spécification utilise un système d'annotations comprenant les
annotations \verb=@Path=, \verb=@GET=, \verb=@POST=, \verb=@Produces=,
\verb=@Consumes= permettant de définir les méthodes à appeler pour une
route et une méthode données.

\subsection{Vert.x}

\emph{Vert.x} est un framework permettant de créer des API Web
Asynchrone.  Il s'inspire grandement de ce qui a été fait dans
\emph{Node.js} avec l'utilisation intensive de callback dans des
requêtes HTTP et les appels vers la base de données. Un point fort de
Vert.x est le fait que les traitements asynchrone soit bien intégré à
l'API. Cela permet d'écrire des traitements de requêtes de manière
relativment déclaratives grâce à des callbacks tout en ayant de bonnes
performances.  À la différence de JAX-RS, il n'y a pas de standard
pouvant donner une bonne confiance dans la pérénnité de Vert.x
d'autant que ce projet créé en 2011 reste relativement jeune.

\section{Proposition actuelle}

\subsection{Apache Camel}

Apache camel est un framework implémentant un ensemble de patterns de
communications par envoi de messages décrits dans l'ouvrage «
Enterprise Integration Patterns » \cite{hohpe2003enterprise}.

Une solution intéressante proposé par ce framework et montré dans la
figure \ref{fig:restdsl} est la définition d'un langage spécifique REST
transposable à la fois en Java et en XML permettant de décrire les
routes et les handlers de traitement des types de requêtes associés à
ces routes.

\begin{figure}
  \ContinuedFloat*
  \begin{lstlisting}[language=java]
rest("/customers/")
  .get("/{id}").to("direct:customerDetail")
  .get("/{id}/orders").to("direct:customerOrders")
  .post("/neworder").to("direct:customerNewOrder");
  \end{lstlisting}
  \caption{REST DSL Java}
\end{figure}

\begin{figure}
  \ContinuedFloat
  \begin{lstlisting}[language=xml]
<rest path="/customers/">
  <get uri="/{id}">
    <to uri="direct:customerDetail"/>
  </get>
  <get uri="/{id}/orders">
    <to uri="direct:customerOrders"/>
  </get>
  <post uri="/neworder">
    <to uri="direct:customerNewOrder"/>
  </post>
</rest>
  \end{lstlisting}
  \caption{REST DSL XML}
  \label{fig:restdsl}
\end{figure}

%% Liens donnés par Nicolas
%% https://www.infoq.com/articles/rest-introduction
%% http://apidoc.adility.com/submission-api
%% http://blog.pilotsystems.net/2012/septembre/les-api-rest

\clearpage
\bibliographystyle{plain}
\bibliography{biblio}

\end{document}
