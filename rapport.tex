\documentclass[a4paper, 11pt]{article}
\usepackage[T1]{fontenc}
\usepackage[utf8]{inputenc}
\usepackage[french]{babel}
\usepackage{listings}

\title{Implémentation d'une API HTTP de style REST pour Apache OFBiz}

\author{Mathieu Lirzin}

\date{\today}

\begin{document}

\maketitle

\begin{abstract}
...
\end{abstract}

\section*{Introduction}

Dans le cadre d'une amélioration du progiciel de gestion intégré
Apache OFBiz nous souhaitons mettre en place une API HTTP basée sur
le style d'architecture REST pour les échanges entre les différents
composants logiciels de type service. Il sera demandé au stagiaire
dans un premier temps de réaliser un état de l'art de l'utilisation de
ce style architectural, puis d'étudier les principes d'architectures
orientées services (SOA) utilisés dans le framework Apache OFBiz.

Après cette phase d'analyse, l'API HTTP basée sur le style REST devra
être implementée dans le framework Apache OFBiz en suivant les
principes SOA.  Ces améliorations devront être réalisées en
interaction avec les projets clients et la communauté Apache OFBiz
afin de permettre leurs intégration dans le projet officiel.

%% \chapter{État de l'art}

\section{Architecture REST}

%% Liens donnés par Nicolas
%% https://www.infoq.com/articles/rest-introduction
%% http://apidoc.adility.com/submission-api
%% https://issues.apache.org/jira/browse/OFBIZ-4274
%% http://blog.pilotsystems.net/2012/septembre/les-api-rest

dslmkfqjmdsklfjqlsdkj \cite{fielding2000architectural}

\bibliographystyle{plain}
\bibliography{biblio.bib}

\end{document}
