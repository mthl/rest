\documentclass{beamer}
\usefonttheme{serif}
\beamertemplatenavigationsymbolsempty

\usepackage[utf8]{inputenc}
\usepackage[T1]{fontenc}
\usepackage[french]{babel}
\usepackage{pgfgantt}
\usepackage{graphicx}
\usepackage{color}
\usepackage{tikz}
\usetikzlibrary{decorations,arrows.meta,shapes,automata,fit,positioning}
\usepackage{tikz-uml}
\tikzumlset{fill class=black!10, fill component=white, fill usecase=black!20, fill object=black!20}
\usepackage{listings}
\lstset{basicstyle=\footnotesize, tabsize=2}
\usepackage{amsmath}

\title{Implémentation d'une \textsc{api http} de style \textsc{rest} pour Apache OFBiz}
\author{Mathieu Lirzin}
\date{14 septembre 2018}

\begin{document}

\begin{frame}
  \titlepage
\end{frame}

\section*{Introduction}

\begin{frame}{Introduction}
  Présentation de mon stage de fin d'études effectué au sein de la
  société Néréide.
  \begin{itemize}
  \item Stage d'une durée de 5 mois (2 Avril $\rightarrow$ 30 Août)
  \item Objectif d'implémenter une \textsc{api} \textsc{http}
  \item Étude du style architectural \textsc{rest} et du framework OFBiz
  \end{itemize}
\end{frame}

\begin{frame}{Plan}
  \tableofcontents
\end{frame}

\AtBeginSection[]
{
  \begin{frame}
    \frametitle{Plan}
    \tableofcontents[currentsection]
  \end{frame}
}

\section{Présentation de l'entreprise}

\begin{frame}{Présentation de l'entreprise}
  \begin{center}
    \includegraphics[height=3cm]{logo_nereide}
  \end{center}
  Néréide est une société de service en logiciel libre fondée en 2004
  spécialisée dans l'intégration du progiciel de gestion intégré
  Apache OFBiz.
  \begin{itemize}
  \item Société coopérative et participative (\textsc{scop})
  \item 11 salariés
  \item Réseau Libre Entreprise
  \end{itemize}
\end{frame}

\begin{frame}{Activités de l'entreprise}
  \begin{itemize}
  \item Dévéloppement spécifique
  \item Maintenance et support applicatif (\textsc{tma})
  \item Administration système
  \item Développement communautaire
  \end{itemize}
\end{frame}

\section{Présentation d'OFBiz}

\begin{frame}{Présentation d'OFBiz}
  \begin{center}
    \includegraphics[height=2cm]{logo_ofbiz}
  \end{center}
  Apache \emph{Open For Business} (OFBiz) est un framework Web ayant
  pour but de faciliter la construction de progiciels de gestion
  intégré créé en 2001.
  \begin{itemize}
  \item Basé sur la plateforme Java
  \item Architecture \textsc{soa}, Modèle de données générique
  \item Polyglotte: Java, Groovy, \textsc{dsl xml}, Freemarker, Javascript.
  \item Publié sous licence Apache 2
  \end{itemize}
\end{frame}

\begin{frame}[fragile]{Architecture d'OFBiz}
  \begin{center}
    \resizebox{0.9\textwidth}{!} {%
    \begin{tikzpicture}
      \tikzstyle{engine}=[draw, thick, rectangle, node distance=10pt, fill=white];
      \node[engine] (entity) {entity engine};
      \node[engine, right=of entity] (service) {service engine};
      \node[engine, right=of service] (screen) {screen engine};

      \tikzstyle{elem}=[rounded rectangle, node distance=7pt];
      \tikzstyle{elemc}=[fill=black!30, elem];
      \tikzstyle{elems}=[fill=red!30, elem];
      \tikzstyle{eleme}=[fill=yellow!30, elem];
      \tikzstyle{elemw}=[fill=blue!30, elem];
      \node[elemw] at (0,7) (webapp1) {webapp};
      \node[elemw, below=of webapp1] (webapp2) {webapp};
      \node[elemc, below=of webapp2] (container1) {container};
      \node[elems, below=of container1] (service1) {service};
      \node[elems, below=of service1] (service2) {service};
      \node[eleme, below=of service2] (entity1) {entity};

      \node[elemw, node distance=40pt, right=of webapp1] (webapp3) {webapp};
      \node[elems, below=of webapp3] (service3) {service};
      \node[elems, below=of service3] (service4) {service};

      \node[eleme, node distance=40pt, right=of webapp3] (entity3) {entity};
      \node[eleme, below=of entity3] (entity4) {entity};
      \node[eleme, below=of entity4] (entity5) {entity};
      \node[elems, below=of entity5] (service5) {service};
      \node[elems, below=of service5] (service6) {service};

      \node[elemw, node distance=60pt, right=of entity3] (webapp4) {webapp};
      \node[elems, below=of webapp4] (service7) {service};
      \node[elems, below=of service7] (service8) {service};
      \node[elems, below=of service8] (service9) {service};
      \node[elems, below=of service9] (service10) {service};

      \tikzstyle{cmpt}=[rectangle, thick, rounded corners, draw=black!80, inner sep=8];
      \node[cmpt, fit=(webapp1)(webapp2)(container1)(service1)(service2)(entity1),
        label=above:\emph{composant A}] (cmpta) {};
      \node[cmpt, fit=(webapp3)(service3)(service4),
        label=above:\emph{composant B}] (cmptb) {};
      \node[cmpt, fit=(service5)(service6)(entity3)(entity4)(entity5),
        label=above:\emph{composant C}] (cmptc) {};
      \node[cmpt, fit=(webapp4)(service7)(service8)(service9)(service10),
        label=above:\emph{plugin X}] (pluginx) {};

      \begin{scope}[on background layer]
        \tikzstyle{bg}=[rectangle, rounded corners=10pt, fill=black!10, inner sep=15]
        \node[bg, fit=(entity) (service) (screen), label=right:\emph{\textsc{Framework}}] (framework) {};
        \node[bg, fit=(cmpta) (cmptb) (cmptc), label=above:\emph{\textsc{Applications}}] (apps) {};
      \end{scope}

      \tikzstyle{dep}=[-{Latex[length=7pt]}, densely dashed]
      \draw[dep] (cmpta.south) -- (entity);
      \draw[dep] (cmpta.south) -- (service);
      \draw[dep] (cmpta.south) -- (screen);
      \draw[dep] (cmptb.south) -- (service);
      \draw[dep] (cmptb.south) -- (screen);
      \draw[dep] (cmptc.south) -- (entity);
      \draw[dep] (cmptc.south) -- (service);
      \draw[dep] (pluginx.south) -- (screen);
      \draw[dep] (pluginx.south) -- (service);
      \draw[dep] (pluginx) -- (cmptc.east);
      \draw[dep] (cmptb) -- (cmpta.east);
    \end{tikzpicture}%
    }
  \end{center}
\end{frame}

\begin{frame}{Architecture du framework}
  \begin{center}
    \begin{tikzpicture}
      \tikzstyle{level}= [draw, thick, rectangle,fill=white, text centered]
      \node[level] (entity) at (0, 0) {entity engine};
      \node[level] (service) at (0, 2) {service engine};
      \node[level] (screen) at (0, 4) {screen engine};
      \tikzstyle{dep}=[-{Latex[length=7pt]}, densely dashed]
      \draw[dep] (screen) -- (service);
      \draw[dep] (service) -- (entity);

      \begin{scope}[on background layer]
        \tikzstyle{bg}=[rectangle, rounded corners=10pt, fill=black!10, inner sep=15]
        \node[bg, fit=(entity) (service) (screen)] (framework) {};
      \end{scope}
    \end{tikzpicture}
  \end{center}
\end{frame}


\begin{frame}[fragile]{Définition d'un composant}
  \scriptsize
\begin{verbatim}
<ofbiz-component name="webtools"
    xmlns:xsi="http://www.w3.org/2001/XMLSchema-instance"
    xsi:noNamespaceSchemaLocation=".../ofbiz-component.xsd">
  <resource-loader name="main" type="component"/>
  <classpath type="dir" location="config"/>

  <entity-resource type="data" reader-name="seed" loader="main"
                   location="data/WebtoolsSecurityPermissionSeedData.xml"/>
  <entity-resource type="data" reader-name="demo" loader="main"
                   location="data/WebtoolsSecurityGroupDemoData.xml"/>
  <service-resource type="model" loader="main"
                    location="servicedef/services.xml"/>
  <webapp name="webtools"
          title="WebTools"
          menu-name="secondary"
          server="default-server"
          location="webapp/webtools"
          base-permission="OFBTOOLS,WEBTOOLS"
          mount-point="/webtools"/>
</ofbiz-component>
\end{verbatim}
\end{frame}

\section{Representational State Transfert}

\begin{frame}{\textsc{rest}}
  Le style architectural \textsc{rest} a été décrit en 2000 par Roy
  Fielding dans sa thèse de doctorat \emph{Architectural styles and
    the design of network-based software architectures}.

  \begin{itemize}
  \item Hypermédia distribué
  \item Adoption facile
  \item Extensible
  \item Passage à l'échelle d'Internet
  \end{itemize}
\end{frame}

\begin{frame}{Dérivation de contraintes}
  \begin{center}
    \begin{tikzpicture}
      \tikzstyle{as}= [draw,rectangle,rounded corners=3pt, font=\scshape]
      \tikzstyle{asc}= [as, fill=black!20]
      \tikzstyle{cst}= [right,scale=0.7]
      \node[as] (o) at (0, 10) [circle,fill] {};
      \node[as] (RR) at (-5, 8) {rr};
      \node[asc] (cache) at (-5, 6) {\$};

      \node[asc] (CS) at (-3, 8) {cs};
      \node[asc] (CSS) at (-3, 6) {css};
      \node[as] (CcacheSS) at (-3, 4) {c\$ss};

      \node[asc] (LS) at (0, 8) {ls};
      \node[as] (LCS) at (0, 6) {lcs};
      \node[as] (LCcacheSS) at (0, 4)  {lc\$ss};

      \node[as] (VM) at (2.8, 8) {vm};
      \node[asc] (COD) at (2.8, 6) {cod};
      \node[as] (LCODCcacheSS) at (2.8, 4) {lcodc\$ss};

      \node[asc] (U) at (5, 8)  {u};
      \node[as] (REST) at (5, 4) {rest};

      \draw[->] (o) edge [bend right=15] node[cst, left]{replicated} (RR);
      \draw[->] (o) edge [bend right=10] node[cst]{separated} (CS);
      \draw[->] (o) edge node[cst]{layered} (LS);
      \draw[->] (o) edge [bend left=2] node[cst]{programmable} (VM);
      \draw[->] (o) edge [bend left=15] node[cst,yshift=10]{uniform interface} (U);
      \draw[->] (RR) edge node[cst]{on-demand} (cache);
      \draw[->] (cache) edge node[cst, left]{cacheable} (CcacheSS);
      \draw[->] (CS) edge node[cst]{stateless} (CSS);
      \draw[->] (CSS) edge node[cst]{reliable} (CcacheSS);
      \draw[->] (LS) edge (LCS);
      \draw[->] (LCS) edge node[cst]{shared} (LCcacheSS);
      \draw[->] (VM) edge (COD);
      \draw[->] (COD) edge node[cst]{extensible} (LCODCcacheSS);
      \draw[->] (U) edge node[cst, text width=3cm, left,xshift=25, yshift=30]{simple visible reusable} (REST);
      \draw[->] (CS.east) edge node[cst, sloped, text
        width=2cm, centered]{intermediate processing} (LCS);
      \draw[->] (CS.east) edge[bend left=15, pos=0.75] node[cst]{mobile} (COD);
      \draw[->] (CcacheSS) edge node[cst,centered, yshift=10]{scalable} (LCcacheSS);
      \draw[->] (LCcacheSS) edge node[cst,left, text width=0.5cm]{multi-org} (LCODCcacheSS);
      \draw[->] (LCODCcacheSS) edge  (REST);
    \end{tikzpicture}
  \end{center}
\end{frame}

\begin{frame}[fragile]{Hypermedia As The Engine Of the Application State}
  \begin{center}
    \footnotesize
    \begin{tikzpicture}[]
      \tikzstyle{state}=[state with output, text width=1.4cm, text
        centered, node distance=1.5cm]
      \tikzstyle{trans}=[font=\ttfamily]
      \tikzstyle{every lower node part}=[font=\ttfamily]
      \node[state,initial] (q_0) at (0, 0) {$q_0$ \nodepart{lower} /};
      \node[state,right=of q_0] (q_1) {$q_1$ \nodepart{lower} /users};
      \node[state,right=of q_1] (q_2) {$q_2$ \nodepart{lower} /users/14};
      \path[->] (q_0) edge [bend left] node[trans,above] {GET /users} (q_1)
      (q_1) edge [bend left] node[trans,above] {GET /users/14} (q_2)
      (q_2) edge [loop below] node[trans,below] {PUT /users/14} ()

    \end{tikzpicture}
  \end{center}
  État de l'application $\neq$ État de la ressource
\end{frame}

\section{Réalisations}

\begin{frame}{Organisation}
  \begin{center}
    \begin{ganttchart}[
        hgrid=true,
        vgrid={*1{dotted}}
      ]{1}{10}
      \gantttitle{Avril}{2}
      \gantttitle{Mai}{2}
      \gantttitle{Juin}{2}
      \gantttitle{Juillet}{2}
      \gantttitle{Août}{2}\\
      \ganttbar{Étude de \textsc{rest}}{1}{2} \\
      \ganttbar{Étude d'OFBiz}{2}{4}  \\
      \ganttbar{Implémentation}{4}{9} \\
      \ganttbar{Rapport}{10}{10}
    \end{ganttchart}
  \end{center}
  \begin{itemize}
  \item Développement en intéraction avec la communauté
  \item Mode pseudo “Agile”
  \end{itemize}
\end{frame}

\begin{frame}{Richardson maturity model}
  \begin{center}
    \begin{tikzpicture}
      \tikzstyle{level}= [rectangle,rounded corners=3pt,text width=6cm,fill=black!20]
      \node[level] (l0) at (0, 0) {\emph{Level 0}: The swamp of POX};
      \node[level] (l1) at (0, 1) {\emph{Level 1}: Resources};
      \node[level] (l2) at (0, 2) {\emph{Level 2}: \textsc{http} verbs};
      \node[level] (l3) at (0, 3) {\emph{Level 3}: Hypermedia controls};
      \node (begin) at (4, 0) {};
      \node (end) at (4, 4) {\textbf{The glory of \textsc{rest}}};
      \draw[->, thick] (begin.south) to (end);
    \end{tikzpicture}
  \end{center}
\end{frame}

\begin{frame}[fragile]{Choix techniques}

  \begin{center}
  \scriptsize
\begin{verbatim}
<request-map uri="LookupGeo">...</request-map>
<request-map uri="createGeo">...</request-map>
<request-map uri="updateGeo">...</request-map>
<request-map uri="deleteGeo">...</request-map>

<request-map uri="geos" method="get">...</request-map>
<request-map uri="geos" method="post">...</request-map>
<request-map uri="geos/{id}" method="update">...</request-map>
<request-map uri="geos/{id}" method="delete">...</request-map>
\end{verbatim}
  \end{center}

  \begin{itemize}
  \item Pas de dépendances à des frameworks externes
  \item Intégration au contrôleur existant
  \end{itemize}
\end{frame}

\begin{frame}[fragile]{Moteur d'écran \textsc{json}}
  OFBiz fournit un \textsc{dsl} pour définir des écrans générique qui
  peuvent être rendu en \textsc{html}, \textsc{xml}, \textsc{csv}, ou
  \textsc{xls}.

  \begin{center}
    \scriptsize
\begin{verbatim}
<form name="ComponentList" type="list" separate-columns="true"
      title="Component List" list-name="componentList" target=""
      odd-row-style="alternate-row" header-row-style="header-row-2"
      default-table-style="basic-table hover-bar">
  <field name="compName">
    <hyperlink description="${compName}"
               target="TestSuiteInfo?compName=${compName}"/>
  </field>
  <field name="rootLocation"><display/></field>
  <field name="enabled"><display/></field>
  <field name="webAppName"><display/></field>
</form>
\end{verbatim}
  \end{center}

  \begin{itemize}
  \item Implémentation partiel du rendu \textsc{json}
  \end{itemize}
\end{frame}

\begin{frame}{Difficultés}
  \begin{itemize}
  \item Complexité du framework
  \item Dette technique importante
    \begin{itemize}
    \item Méthode \texttt{doRequest} de 800 lignes
    \item Mutations et tests à \emph{null}
    \end{itemize}
  \item Traitement spécial des \textsc{uri} avec un chemin composé
    (ex. \texttt{/foo/bar/baz})
  \item Latence des retours de la communauté
  \end{itemize}
\end{frame}

\begin{frame}{Ré-usinage}
  \begin{itemize}
  \item Découpage de la méthode \texttt{doRequest}
  \item Utilisation de lambda et l'\textsc{api} des stream
  \item Ajout de tests unitaires
  \end{itemize}
\end{frame}

\section*{Conclusion}

\begin{frame}{Conclusion}
  \begin{block}{}
    Ce stage a rempli mon objectif de découvrir le développement de
    logiciel libre dans un cadre professionnel.
  \end{block}

  \begin{block}{}
    Les premiers développements réalisés ont été intégrés dans OFBiz
    et sont en mesure de fournir une ligne directrice pour une reprise
    du projet.
  \end{block}

  \begin{block}{}
    Mise en perspective de la \emph{technique} avec des problématiques
    fonctionnelles.
  \end{block}
\end{frame}

\begin{frame}{Fin}
  Merci de votre attention. Questions?
\end{frame}

\end{document}
