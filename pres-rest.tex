\documentclass{beamer}
\usefonttheme{serif}
\beamertemplatenavigationsymbolsempty

\usepackage[utf8]{inputenc}
\usepackage[T1]{fontenc}
\usepackage[french]{babel}
\usepackage{graphicx}
\usepackage[x11names, rgb]{xcolor}
\usepackage{tikz}
\usetikzlibrary{snakes,arrows,shapes,automata}
\usepackage{amsmath}

\title{Representational State Transfert}
\author{Mathieu Lirzin}
\institute{Néréide}
\date{16 avril 2018}
\begin{document}

\begin{frame}
  \titlepage
\end{frame}

\begin{frame}{Introduction}
  \emph{Representational State Transfert} (REST)
  \begin{itemize}
  \item Style architectural définit par Roy Fielding
  \item Modèle idéalisé du \emph{World Wide Web}
  \item Une API utilisant le style REST est dîte \emph{RESTful}
  \item Buzzword!
  \end{itemize}

\end{frame}

\section{Aperçu conceptuel}

\subsection{Définitions}

\begin{frame}{Architecture logicielle}
  \begin{block}{Définition}
    Une abstraction des élements s'exécutant dans une phase de
    l'exécution d'un système.  Peut être décomposée en différents
    niveaux d'abstraction et différentes phases.
  \end{block}

  \begin{block}{}
    Une architecture logicielle est une configuration des élements
    suivants:
    \begin{itemize}
    \item Composants
      %% Un composant est une unité abstraite d'instructions
      %% logicielles et d'une état internet qui peut être modifié via
      %% son interface
    \item Connecteurs
      %% Mécanisme abstrait qui médiatise la communication,
      %% coordination et coopération entre différents composants.
    \item Données
      %% Une donnée est un élément d'information qui est transmises
      %% entre les composants par un connecteur
    \end{itemize}
  \end{block}
\end{frame}

\begin{frame}{Style architectural}
  \begin{block}{Définition}
    Un ensemble de contraintes architecturals qui restreignent les rôles
    et possibilités des différents éléments architecturaux.
  \end{block}

  \begin{block}{}
    Une architecture est une instance d'une combinaison de plusieurs
    styles archecturaux qui peuvent être dérivés les uns des autres
  \end{block}
\end{frame}

\section{Contexte Web}

\subsection{Objectifs}

\begin{frame}{Analyse des besoins}
  Le style architectural REST doit correspondre de manière générique
  aux besoins du Web.

  \begin{itemize}
  \item Système d'information partagé à l'échelle d'Internet
  \item Adoption facile
  \item Support de l'hétérogénéité des terminaux
  \item Tolère la participation éphémère
  \item Extensible
  \end{itemize}
\end{frame}

\begin{frame}{Propriétés}
  \begin{itemize}
  \item Performance
  \item Passage à l'échelle
  \item Simplicité
  \item Modifiable
  \item Visibilité
  \item Portabilité
  \item Robustesse
  \end{itemize}
\end{frame}

\begin{frame}{Contraintes}
  \begin{itemize}
  \item Client-Serveur
  \item Sans connexion
  \item Possible mise en cache
  \item En couches
  \item Code à la demande (optionel)
  \item Interface uniforme
  \end{itemize}
\end{frame}

\begin{frame}{Dérivation de contraintes}
  \begin{center}
    \begin{tikzpicture}
      \tikzstyle{as}= [draw,rectangle,rounded corners=3pt]
      \tikzstyle{asc}= [draw,rectangle,rounded corners=3pt, fill=blue!20]
      \tikzstyle{cst}= [right,scale=0.7]
      \node[as] (o) at (0, 10) [circle,fill] {};
      \node[as] (RR) at (-5, 8) {RR};
      \node[asc] (cache) at (-5, 6) {\$};

      \node[asc] (CS) at (-3, 8) {CS};
      \node[asc] (CSS) at (-3, 6) {CSS};
      \node[as] (CcacheSS) at (-3, 4) {C\$SS};

      \node[asc] (LS) at (0, 8) {LS};
      \node[as] (LCS) at (0, 6) {LCS};
      \node[as] (LCcacheSS) at (0, 4)  {LC\$SS};

      \node[as] (VM) at (2.8, 8) {VM};
      \node[asc] (COD) at (2.8, 6) {COD};
      \node[as] (LCODCcacheSS) at (2.8, 4) {LCODC\$SS};

      \node[asc] (U) at (5, 8)  {U};
      \node[as] (REST) at (5, 4) {REST};

      \draw[->] (o) edge [bend right=15] node[cst, left]{replicated} (RR);
      \draw[->] (o) edge [bend right=10] node[cst]{separated} (CS);
      \draw[->] (o) edge node[cst]{layered} (LS);
      \draw[->] (o) edge [bend left=2] node[cst]{programmable} (VM);
      \draw[->] (o) edge [bend left=15] node[cst,yshift=10]{uniform interface} (U);
      \draw[->] (RR) edge node[cst]{on-demand} (cache);
      \draw[->] (cache) edge node[cst, left]{cacheable} (CcacheSS);
      \draw[->] (CS) edge node[cst]{stateless} (CSS);
      \draw[->] (CSS) edge node[cst]{reliable} (CcacheSS);
      \draw[->] (LS) edge (LCS);
      \draw[->] (LCS) edge node[cst]{shared} (LCcacheSS);
      \draw[->] (VM) edge (COD);
      \draw[->] (COD) edge node[cst]{extensible} (LCODCcacheSS);
      \draw[->] (U) edge node[cst, text width=3cm, left,xshift=25, yshift=30]{simple visible reusable} (REST);
      \draw[->] (CS.east) edge node[cst, sloped, text
        width=2cm, centered]{intermediate processing} (LCS);
      \draw[->] (CS.east) edge[bend left=15, pos=0.75] node[cst]{mobile} (COD);
      \draw[->] (CcacheSS) edge node[cst,centered, yshift=10]{scalable} (LCcacheSS);
      \draw[->] (LCcacheSS) edge node[cst,left, text width=0.5cm]{multi-org} (LCODCcacheSS);
      \draw[->] (LCODCcacheSS) edge  (REST);
    \end{tikzpicture}
  \end{center}
\end{frame}

\begin{frame}{Architecture Web}
  Le \emph{World Wide Web} est une instance du style architectural
  REST.

  \begin{description}
  \item [Composants] User Agent (Navigateur), Gateway, Proxy
  \item [Connecteurs] client (XMLHttpRequest, fetch), serveur (httpd), cache, DNS, TLS
  \item [Données] Ressource, URI, HTML, image, meta-donnée, media-type
  \end{description}
\end{frame}

\begin{frame}{Données}
  Les ressources sont identifiées par des URIs qui permettent
  d'accéder aux différentes represéntations d'une ressource.

  \begin{center}
    \begin{tikzpicture}
      \tikzstyle{n}= [draw,rectangle,rounded corners=3pt]
      \node[n, circle] (c) at (0, 0) {Client};
      \node[n] (rep) at (4, 0) {Représentation};
      \node[n] (res) at (9, 0) {Ressource};
      \draw[->] (c) edge [bend left] node [above] {envoit} (rep);
      \draw[->] (rep) edge [bend left] node [below] {recoit} (c);
      \draw[<->] (rep) edge node [above] {met à jour} (res);
    \end{tikzpicture}
  \end{center}

  Les représentations sont caractérisées par un \emph{media-type}
  définit dans le header de la réponse.
\end{frame}

\begin{frame}{Requêtes auto-descriptives}
  Les verbes correspondants aux méthodes du procotole HTTP sont
  porteurs d'une sémantique fixe:
  \begin{center}
    \begin{tabular}{lcc}
      Méthodes & Idempotent & Sans effet \\
      \hline
      \verb#OPTIONS# & oui  &  oui \\
      \verb#GET# & oui &  oui \\
      \verb#HEAD# & oui & oui \\
      \verb#PUT# & oui & non \\
      \verb#POST# & non & non \\
      \verb#DELETE# & oui & non \\
      \verb#PATCH# & non & non \\
      \verb#TRACE# & non & oui \\
      \verb#CONNECT# & non & non \\
    \end{tabular}
  \end{center}
\end{frame}

\begin{frame}{Réponses auto-descriptives}
  Catégories de codes d'erreurs basées sur des nombres à 3 chiffres
  avec le chiffre des dizaines et des unités servant à raffiner la
  catégorie.

  \begin{description}
  \item [100] En attente
  \item [200] Succès
  \item [300] Redirection
  \item [400] Non autorisé ou indisponible
  \item [500] Erreur coté serveur
  \end{description}
\end{frame}

\begin{frame}{Richardson maturity model}
  \begin{center}
    \begin{tikzpicture}
      \tikzstyle{level}= [draw,rectangle,rounded corners=3pt,text width=6cm,fill=blue!20]
      \node[level] (l0) at (0, 0) {Level 0: The swamp of POX};
      \node[level] (l1) at (0, 1) {Level 1: Resources};
      \node[level] (l2) at (0, 2) {Level 2: HTTP verbs};
      \node[level] (l3) at (0, 3) {Level 3: Hypermedia controls};
      \node (begin) at (4, 0) {};
      \node (end) at (4, 4) {\textbf{The glory of REST}};
      \draw[->] (begin.south) to (end);
    \end{tikzpicture}
  \end{center}
\end{frame}

\begin{frame}{Hypermédia}
  Mécanisme simple et générique liant les ressources entre
  elles au moyen de liens. Cela permet:
  \begin{itemize}
  \item Découvrabilité des ressources
  \item Faible couplage
  \item Maintenabilité
  \end{itemize}

  Pour supporter l'hypermédia, le format des données doit pouvoir
  contenir des liens vers d'autres URI ainsi que les actions possibles
  sur ces URI.
\end{frame}

\begin{frame}[fragile]{Format \verb#HTML#}
  \scriptsize
\begin{verbatim}
<html>
  <head>
    <script src="hello.js"></script>
    <link rel="stylesheet" type="text/css" href="theme.css">
  </head>
  <body>
    <img src="example.com/img.png">
    ...
    <form action="" method="post">
      <fieldset>
        <legend>Titre</legend>
        <input type="radio" name="radio" id="radio">
        <label for="radio">Cliquez moi</label>
      </fieldset>
    </form>
  </body>
</html>
\end{verbatim}
\end{frame}

\begin{frame}[fragile]{Format \verb#hal+json#}
\scriptsize
\begin{verbatim}
{
  "_links": {
    "self": { "href": "http://example.com/api/book/hal-cookbook" },
    "next": { "href": "http://example.com/api/book/hal-case-study" },
    "prev": { "href": "http://example.com/api/book/json-and-beyond" },
    "first": { "href": "http://example.com/api/book/catalog" },
    "last": { "href": "http://example.com/api/book/upcoming-books" }
  },
 "_embedded": {
    "author": {
      "_links": {
        "self": { "href": "http://author-example.com" }
      },
      "id": "shahadat",
      "name": "Shahadat Hossain Khan"
    }
  },
  "id": "hal-cookbook",
  "name": "HAL Cookbook"
}
\end{verbatim}
\end{frame}

\begin{frame}[fragile]{Hypermedia As The Engine Of the Application State}
  \begin{center}
    \resizebox{1.0\textwidth}{!} {%
      \begin{tikzpicture}[shorten >=1pt,node distance=2cm,auto]
        \tikzstyle{state}=[state with output]
        \node[state,initial] (q_0) at (0, 0) {$q_0$ \nodepart{lower} $/$};
        \node[state] (q_1) at (0, 3) {$q_1$ \nodepart{lower} $/users$};
        \node[state] (q_2) at (5, 3) {$q_2$ \nodepart{lower} $/users/14$};
        \node[state] (q_3) at (5, 0) {$q_3$ \nodepart{lower} $/users$};
        \node[state] (q_4) at (9, 1.5) {$q_4$ \nodepart{lower} $/users/14$};
        \path[->] (q_0) edge node {\texttt{POST /users}} (q_1)
                        edge [loop below] node {\texttt{GET /}} ()
                        edge [bend left] node {\texttt{GET /users}} (q_3)
                  (q_1) edge [bend left] node {\texttt{GET /users/14}} (q_2)
                        edge [loop above] node {\texttt{GET /users}}
                        ()
                  (q_2)  edge [bend left] node {\texttt{GET /users}} (q_1)
                        edge[bend left] node {\texttt{PUT /users/14}} (q_4)
                  (q_3) edge [bend left] node {\texttt{GET /}} (q_0)
                  (q_4) edge [bend left] node {\texttt{DELETE /users/14}} (q_3)
      \end{tikzpicture}%
    }
  \end{center}
\end{frame}

\begin{frame}{Le challenge sémantique}
  Pour vraiment atteindre son objectif d'interopérabilité entre
  différentes API, un problème ouvert est de définir des sémantiques
  associées aux ressources.
\end{frame}

\begin{frame}{Fin}
  Merci de votre attention. Questions?
\end{frame}

\end{document}
